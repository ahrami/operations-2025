\documentclass[a4paper,14pt]{extarticle}

\usepackage[onehalfspacing,marginsgeneric,midindent,nosectionname,nolistindent]{config}
\addbibresource{sources.bib}
\graphicspath{ {./images/} }

\begin{document}

\newgeometry{left=20mm, top=20mm, right=20mm, bottom=20mm, nohead, nofoot}
\begin{titlepage}
\begin{center}

\vspace{35mm}

\textbf{\large Конспект}

\textbf{\textit{\large Название работы}}

\vspace{20mm}

\vfill

{\the\year{} г.}
\end{center}
\end{titlepage}

\restoregeometry
\addtocounter{page}{1}


\newpage
\tableofcontents

\usection{Вопросы}

\begin{enumerate}
    \item Матричная игра. Максиминные и минимаксные стратегии. Равновесия.
        (\textbf{\S 1.1, \S 1.2, \S 1.3 в \cite{gametheory-2012}})
    \item Необходимое и достаточное условие существования оптимальных стратегий в матричной игре.
        (\textbf{Теорема в \S 1.3, стр. 20 в \cite{gametheory-2012}})
    \item Смешанное расширение матричной игры.
        (\textbf{\S 1.4. в \cite{gametheory-2012}})
    \item Свойства оптимальных смешанных стратегий.
        (\textbf{\S 1.7 в \cite{gametheory-2012}})
    \item Теорема о существовании ситуации равновесия в смешанных стратегиях.
        (\textbf{\S 1.6 в \cite{gametheory-2012}})
    \item Теорема о доминировании для матричных игр.
        (\textbf{\S 1.8 в \cite{gametheory-2012}})
    \item Понятие игры в нормальной форме. Равновесие по Нэшу. Примеры.
        (\textbf{\S 3.1, \S 3.2 в \cite{gametheory-2012}})
    \item Доказательство существования ситуации равновесия по Нэшу в смешанных стратегиях для конечных игр.
        (\textbf{\S 3.4, \S 3.5 в \cite{gametheory-2012}})
    \item Кооперативные игры. Характеристическая функция. Дележи.
        (\textbf{\S 3.11 в \cite{gametheory-2012}})
    \item Доминирование дележей.
        (\textbf{\S 3.11 в \cite{gametheory-2012}})
    \item С-ядро игры. Необходимое и достаточное условие непустоты С-ядра.
        (\textbf{\S 3.12 в \cite{gametheory-2012}})
    \item НМ-решение.
        (\textbf{\S 3.12 в \cite{gametheory-2012}})
    \item Вектор Шепли.
        (\textbf{\S 3.13 в \cite{gametheory-2012}})
    \item Ромбовидная иерархическая игра.
        (\textbf{\S 4.5, стр. 201 (4.5.4) в \cite{gametheory-2012}})
    \item Абсолютное равновесие по Нэшу. Примеры.
        (\textbf{\S 4.2 в \cite{gametheory-2012}})
    \item Теорема о существовании абсолютного равновесия по Нэшу в конечношаговой игре с полной информацией,
        (\textbf{\S 4.2 в \cite{gametheory-2012}})
    \item Примеры различных типов равновесий по Нэшу в играх с полной информацией.
    \item Древовидные иерархические игры.
        (\textbf{\S 4.4 в \cite{gametheory-2012}})
    \item Существование равновесия по Нэшу в смешанных стратегиях в играх с конечным числом стратегий.
    \item Равновесия по Штакельбергу (сравнение с равновесием по Нэшу для иерархической древовидной игры)
        (\textbf{\S 3.2, стр. 128 (3.2.5) в \cite{gametheory-2012}}, TODO)
    \item Пример реализации принципа максимума.
    \item Многошаговые игры с полной информацией.
        (\textbf{\S 4.1 в \cite{gametheory-2012}})
    \item Итеративный метод решения матричной игры (метод Брауна-Робинсон)
        (\textbf{\S 1.10.1 в \cite{gametheory-2012}})
    \item Парадокс Браеса.
        (\textbf{\cite{wiki-braess-ru}}, TODO) % TODO
    \item Простейшая задача управления запасами.
        (\textbf{\S 8.3 в \cite{vagner-2-1983}})
    \item Задача замены оборудования.
        (\textbf{\S 10.6 в \cite{vagner-2-1983}})
    \item Примеры задач динамического программирования.
    \item Уравнение Беллмана для детерминированного многошагового процесса принятия решений.
    \item Принцип оптимальности Беллмана. Решение задачи об оптимальном быстродействии.
    \item Непрерывные задачи динамического программирования.
    \item Связь динамического программирования с принципом максимума Л.С. Понтрягина.
    \item Пример решения задачи оптимального управления с использованием принципа максимума.
    \item Дискретная задача на быстродействие.
    \item Задачи динамического программирования на марковских цепях (пример --- фабрикант кукол).
\end{enumerate}

\custombibliography

\end{document}
