\documentclass[a4paper,14pt]{extarticle}

\usepackage[onehalfspacing,marginsgeneric,midindent,nosectionname,nolistindent]{config}
\addbibresource{sources.bib}
\graphicspath{ {./images/} }

\begin{document}

\newgeometry{left=20mm, top=20mm, right=20mm, bottom=20mm, nohead, nofoot}
\begin{titlepage}
\begin{center}

\vspace{35mm}

\textbf{\large Конспект}

\textbf{\textit{\large Методы и модели исследования операций}}

\vspace{20mm}

\vfill

{\the\year{} г.}
\end{center}
\end{titlepage}

\restoregeometry
\addtocounter{page}{1}


\newpage
\tableofcontents

\usection{Вопросы}

\begin{enumerate}
    \item Матричная игра. Максиминные и минимаксные стратегии. Равновесия.
        (\textbf{\S 1.1, \S 1.2, \S 1.3 в \cite{gametheory-2012}})
    \item Необходимое и достаточное условие существования оптимальных стратегий в матричной игре.
        (\textbf{Теорема в \S 1.3, стр. 20 в \cite{gametheory-2012}})
    \item Смешанное расширение матричной игры.
        (\textbf{\S 1.4. в \cite{gametheory-2012}})
    \item Свойства оптимальных смешанных стратегий.
        (\textbf{\S 1.7 в \cite{gametheory-2012}})
    \item Теорема о существовании ситуации равновесия в смешанных стратегиях.
        (\textbf{\S 1.6 в \cite{gametheory-2012}})
    \item Теорема о доминировании для матричных игр.
        (\textbf{\S 1.8 в \cite{gametheory-2012}})
    \item Понятие игры в нормальной форме. Равновесие по Нэшу. Примеры.
        (\textbf{\S 3.1, \S 3.2 в \cite{gametheory-2012}})
    \item \label{nash} Доказательство существования ситуации равновесия по Нэшу в смешанных стратегиях для конечных игр.
        (\textbf{\S 3.4, \S 3.5, стр. 136 (3.5.6) в \cite{gametheory-2012}})
    \item Существование равновесия по Нэшу в смешанных стратегиях в играх с конечным числом стратегий.
        (Probably same as \ref{nash})
    \item Кооперативные игры. Характеристическая функция. Дележи.
        (\textbf{\S 3.11 в \cite{gametheory-2012} и 20 в \cite{sharshukov-xyz}})
    \item Доминирование дележей.
        (\textbf{\S 3.11 в \cite{gametheory-2012} и 20 в \cite{sharshukov-xyz}})
    \item С-ядро игры. Необходимое и достаточное условие непустоты С-ядра.
        (\textbf{\S 3.12 в \cite{gametheory-2012} и 21 в \cite{sharshukov-xyz}})
    \item НМ-решение.
        (\textbf{\S 3.12 в \cite{gametheory-2012} и 21 в \cite{sharshukov-xyz}})
    \item Вектор Шепли.
        (\textbf{\S 3.13 в \cite{gametheory-2012} и 22 в \cite{sharshukov-xyz}})
    \item Многошаговые игры с полной информацией.
        (\textbf{\S 4.1 в \cite{gametheory-2012}})
    \item Абсолютное равновесие по Нэшу. Примеры.
        (\textbf{\S 4.2 в \cite{gametheory-2012}})
    \item Теорема о существовании абсолютного равновесия по Нэшу в конечношаговой игре с полной информацией,
        (\textbf{\S 4.2, стр.191 в \cite{gametheory-2012}})
    \item Примеры различных типов равновесий по Нэшу в играх с полной информацией.
        (TODO)
    \item Древовидные иерархические игры.
        (\textbf{\S 4.4 в \cite{gametheory-2012}})
    \item Ромбовидная иерархическая игра.
        (\textbf{\S 4.5, стр. 201 (4.5.4) в \cite{gametheory-2012}})
    \item Равновесия по Штакельбергу (сравнение с равновесием по Нэшу для иерархической древовидной игры)
        (\textbf{\S 3.2, стр. 128 (3.2.5) в \cite{gametheory-2012}}, TODO)
    \item Итеративный метод решения матричной игры (метод Брауна-Робинсон)
        (\textbf{\S 1.10.1 в \cite{gametheory-2012}})
    \item Парадокс Браеса.
        (\textbf{\cite{wiki-braess-ru}}, TODO)
    \item Простейшая задача управления запасами.
        (\textbf{\S 8.3 в \cite{vagner-2-1983} и \S 11 в \cite{taha-2002}})
    \item Задача замены оборудования.
        (\textbf{\S 10.6 в \cite{vagner-2-1983} и \S 10.3.3 в \cite{taha-2002}})
    \item Примеры задач динамического программирования.
        (\textbf{\S 10.3 в \cite{taha-2002} и 23 в \cite{sharshukov-xyz}})
    \item Уравнение Беллмана для детерминированного многошагового процесса принятия решений.
        (\textbf{24 в \cite{sharshukov-xyz} и \cite{kuznetsov-dp}})
    \item \label{speed} Принцип оптимальности Беллмана. Решение задачи об оптимальном быстродействии.
        (\textbf{25 в \cite{sharshukov-xyz} и \cite{kuznetsov-dp}})
    \item Дискретная задача на быстродействие.
        (Probably same as \ref{speed})
    \item Задачи динамического программирования на марковских цепях (пример --- фабрикант кукол).
        (TODO)
    \item Непрерывные задачи динамического программирования.
        (\textbf{\S 6.1 в \cite{gametheory-2012} и 26 в \cite{sharshukov-xyz}})
    \item Связь динамического программирования с принципом максимума Л.С. Понтрягина.
        (\textbf{\S 6.2 в \cite{gametheory-2012}})
    \item \label{maximum} Пример решения задачи оптимального управления с использованием принципа максимума.
        (\textbf{\S 6.2 в \cite{gametheory-2012}})
    \item Пример реализации принципа максимума.
        (Probably same as \ref{maximum})
\end{enumerate}

\custombibliography

\end{document}
